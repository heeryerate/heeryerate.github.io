%!TEX program = xelatex
%%%%%%%%%%%%%%%%%%%%%%%%%%%%%%%%%%%%%%%%%
% Plasmati Graduate CV
% LaTeX Template
% Version 1.0 (24/3/13)
%
% This template has been downloaded from:
% http://www.LaTeXTemplates.com
%
% Original author:
% Alessandro Plasmati (alessandro.plasmati@gmail.com)
%
% License:
% CC BY-NC-SA 3.0 (http://creativecommons.org/licenses/by-nc-sa/3.0/)
%
% Important note:
% This template needs to be compiled with XeLaTeX.
% The main document font is called Fontin and can be downloaded for free
% from here: http://www.exljbris.com/fontin.html
%
%%%%%%%%%%%%%%%%%%%%%%%%%%%%%%%%%%%%%%%%%

%----------------------------------------------------------------------------------------
%   PACKAGES AND OTHER DOCUMENT CONFIGURATIONS
%----------------------------------------------------------------------------------------

\documentclass[letters,11pt]{article} % Default font size and paper size

% \usepackage[margin=1.5cm]{geometry}
\usepackage[margin = 1.5cm]{geometry}
\geometry{bottom=25mm}
\usepackage{longtable}
\usepackage{booktabs}
\usepackage{tabularx}
\usepackage{ltablex} 
\usepackage{amssymb}
\usepackage{array}

\usepackage{fontspec} % For loading fonts
\defaultfontfeatures{Mapping=tex-text}
% \setmainfont[SmallCapsFont = Fontin SmallCaps]{Fontin} % Main document font

\usepackage{xunicode,xltxtra,url,parskip} % Formatting packages

\usepackage[usenames,dvipsnames]{xcolor} % Required for specifying custom colors

% \usepackage[big]{layaureo} 
% Margin formatting of the A4 page, an alternative to layaureo can be 
% \usepackage{fullpage}
% To reduce the height of the top margin uncomment: 
% \addtolength{\voffset}{-1.3cm}

\usepackage{hyperref} % Required for adding links   and customizing them
\definecolor{linkcolour}{rgb}{0,0.2,0.6} % Link color
\hypersetup{colorlinks,breaklinks,urlcolor=linkcolour,linkcolor=linkcolour} % Set link colors throughout the document

\usepackage{titlesec} % Used to customize the \section command
\titleformat{\section}{\Large\scshape\raggedright}{}{0em}{}[\titlerule] % Text formatting of sections
\titlespacing{\section}{0pt}{0pt}{0pt} % Spacing around sections

\usepackage[english]{babel}
\usepackage{fancyhdr}
\usepackage{lastpage}

% \pagestyle{plain}
\pagestyle{fancy}
\fancyhf{}
\renewcommand{\headrulewidth}{0pt}
\renewcommand{\footrulewidth}{1pt}
\newcommand{\lst}[1]{\quad\footnotesize{$\blacktriangleright$ #1.}}

\usepackage{marvosym}


\begin{document}

% \pagestyle{roman} % Removes page numbering
% \pagenumbering{roman}

\lfoot{Curriculum Vitae: Xi He, Ph.D.}
\rfoot{Page \thepage \hspace{1pt} of \pageref{LastPage}}
\font\fb=''[cmr10]'' % Change the font of the \LaTeX command under the skills section

%----------------------------------------------------------------------------------------
%   NAME AND CONTACT INFORMATION
%----------------------------------------------------------------------------------------

\begin{table}
    \centering
    \begin{longtable}{c}    
\par{\centering{\Huge \textbf{Xi} \textbf{He}}\par }\\ % Your name
\par{\centering{837 Cedar Hill Drive, Allentown, PA, 18109}\par}\\ % Your name
\par{\centering{(484)633-8040 \quad \Email~~xih314@lehigh.edu\quad \href{http://xihey.com}{http:{\scriptsize{//}}xihey.com}}\par} \\% Your name
% \par{\centering{(484)633-8040}\par}\\ % Your name
% \par{\centering{\href{http://xihey.com}{http://xihey.com}}\par} % Your name
\end{longtable}
\end{table}

%----------------------------------------------------------------------------------------
%   EDUCATION
%----------------------------------------------------------------------------------------

\section{Education}
\begin{longtable}{>{\centering}p{3.3cm}|p{14cm}}    
Aug 14' - Present &  \textbf{Ph.D. Candidate in Lehigh University, Bethlehem, PA, USA}   \\
&Major...............:~~Industrial and System Engineering \\
&Current advisor:~~\href{http://mtakac.com/}{Prof. Martin Takáč} \\
%------------------------------------------------
Aug 12' - May 14'&  \textbf{Master of Science in Nankai University, Tianjin, China} \\
&Major...............:~~Computational Mathematics\\
% &Advisor............:~~\,Prof. Qingzhi Yang \\
%------------------------------------------------
Sep 08' - Jun 12' & \normalsize\textbf{Bachelor of Science in Nankai University, Tianjin, China} \\
&Major...............:~~Mathematics \\
\end{longtable}

% \section{Research Interests}
% \begin{longtable}{p{17.7cm}}
%     \qquad My research focuses on large-scale optimization algorithms and its applications
%     in machine learning, more specifically, large-scale nonlinear optimization,
%     stochastic gradient methods, statistical learning and high performance computing.
% \end{longtable}

\section{Working Experience}
\begin{longtable}{>{\centering}p{3.35cm}|p{14cm}}
    \small{\textbf{Participant}}& \textit{Intel Corporation, Santa Clara, CA, USA}\\\
    \small{Aug 15' - Present}&\textbf{Large-scale Distributed Optimization in Deep Neural Networks}\\ 
    & \lst{Implemented various of standard approached to training deep learning model}\\
    &\lst {Applied distributed high performance computing technique to accelerate training rate}\\
    \small{\textbf{Research Assistant}}& \textit{Department of Industrial and Systems Engineering, Lehigh University}\\\
    \small{Nov 15 - Feb 16'}&\textbf{Dual Free Mini-batch SDCA with adaptive probabilities}\\ 
    & \lst{Derived optimal probability distribution for dual free SDCA by exploring sub-optimality}\\
        &\lst {Developed unbiased non-uniform mini-batch sampling techniques to improve performance}\\
 & \lst{Guaranteed better complexity bound and convergence rate for the adaptive algorithm}\\
    % &\textbf{Asynchronous CoCoA}\\ 
    % & \lst{Propose asynchronous distributed algorithms for empirical minimization problem}\\
    \small{\textbf{Internship}}& \textit{Predictive Anytics and Monitoring, Siemens Corporation, Princeton, NJ, USA}\\\
    \small{June 15' - Sep 15'}&\textbf{Deep Learning via Hessian-Free Approach}\\
    & \lst{Proposed new algorithm which made use of its approximated local Hessian Matrix information}\\
    &\lst{Guaranteed to reach local optimality instead of sticking at critical point}\\
    % &\lst{Shared better and more stable performance}\\
    & \textbf{Estimating Large-Loss Probability in Credit Portfolio Risk}   \\
    &  \lst{Derived optimal risk loading coefficients by fully using dependency information among obligors}\\
    &  \lst{Estimated large-loss probability of a portfolio by normal copula model and important sampling}\\
    \small{\textbf{Research Assistant}}& \textit{Department of Industrial and Systems Engineering, Lehigh University}\\\
    \small{Sep 14' - May 15'}&\textbf{Asynchronous CoCoA}\\ 
    & \lst{Proposed asynchronous distributed algorithms for empirical minimization problem}\\   
     & \lst{Analyzed communication efficient protocol to achieve better speed-up}\\
    % & \textbf{RCD method on large-scale optimization problems with linear constraints}   \\
    % &  \lst{}\\

    % \small{\textbf{Research Assistant}}& \textit{Department of Mathematics, Nankai University}\\
    % \small{June 15' - Sep 15'}&\textbf{Nonnegative Square (Rectangular) Tensor}\\
    % & \lst{Classify nonnegative tensors under different properties.}\\
    % & \lst{Develop fast algorithm to find spectral radius of nonnegative square tensor and implement it by Matlab and Mathematica.}\\
    % &\lst{Apply tensor theory to hypergraph and polynomial optimization problems.}\\

    \small{\textbf{Teaching Assistant}}& \textit{Department of Industrial and Systems Engineering, Lehigh University}\\
    \small{Sep 14' - May 15'}&\textbf{Applied Engineering Statistics}\\ 
    % \small{Sep 13' - Jan 14' }&\textbf{Theory of Optimization}\\
    % \small{Sep 13' - Jan 14' }&\textbf{Linear algebra}\\
\end{longtable}

\section{Publications}
\begin{longtable}{>{\centering}p{3.3cm}|p{14cm}}
    Conference &
     \textbf{[1] Dual Free Adaptive Mini-Batch SDCA for Empirical Risk Minimization}, with Martin Takáč. Under Reviewed by ICML 2016.\\
    &\textbf{[2] Dual Free SDCA for Empirical Risk Minimization with Adaptive Probabilities}, with Martin Takáč. Accepted by NIPS 2015.\\
    &\textbf{[3] Estimating Portfolio Loss Probabilities with Optimal Risk Loading Coefficients and Fixed Dependency among Obligors,} with Amit Chakraborty, Ioannis Akrotirianakis. \\
    % &\textbf{[3] Modeling Probability of Rare Catastrophic Failure in Complex Systems Using a Copula-based Approach,} with Amit Chakraborty, Ioannis Akrotirianakis. \\
    &\textbf{[4] Exploiting negative curvature in deep learning optimization problems,} with Ioannis Akrotirianakis, Amit Chakraborty.\\
    &\textbf{[5] Asynchronous Distributed Stochastic dual (Block) Coordinate Descent Methods,} with Martin Takáč. \\
    \pagebreak
    &\textbf{[6] Coordinate Descent Methods for Linearly Constrained Optimization,} with Martin Takáč.\\
    Journal &\textbf{[7] A Method with Parameter for Solving the Spectral Radius of Nonnegative Tensor,} with Yiyong Li, Qingzhi Yang. Under Review.
\end{longtable}

\section{Computing Skills}
\begin{longtable}{>{\centering}p{3.3cm}|p{14cm}}
    Programming & \textsc{C\texttt{++} (MPI, OpenMP)}, \textsc{Matlab}, \textsc{R}, \textsc{Python (Spark)}, \textsc{Mathematica}\\
    Optimization & AMPL, CPLEX, MOSEK, Gurobi \\
    {Others} &  \textsc{Shell Script}, \LaTeX, Mac OS, Linus, Windows
\end{longtable}

% \section{Working and Volunteer Experience}
% \begin{longtable}{>{\centering}p{3.3cm}|p{14cm}}
%     Jun 15' - Sep 15' \,&\textbf{Research Assistant,} \emph{Predictive Analytics
%     Business Analytics and Monitoring,} Siemens Corporation, Corporate Technology, Princeton, NJ. \\
%     & \lst{Portfolio Credit Risk and Deep Learning.}\\
%     Sep 14' - Jun 15' \,&\textbf{Teaching Assistant,} \emph{Department of Industrial and Systems Engineering,} Lehigh University, Bethlehem, PA. \\
%     & \lst{Applied Engineering Statistics (Fall 2014 \& Spring 2015).}\\
%     Sep 13' - Jan 14' \,& \textbf{Teaching Assistant,} \emph{Department of Mathematics,} Nankai University, P.R.China. \\
%     & \lst{Theory of Optimization (Spring 2014) \& Linear algebra (Fall 2013).}\\
% \end{longtable}

\section{Selected courses and Projects}
\begin{longtable}{>{\centering}p{3.3cm}|p{14cm}}
    Spring 16'& \textbf{Optimization in Machine Learning,} Lehigh University.\\ 
    Fall 15'& \textbf{Massive Data Mining,} Lehigh University.\\ 
    &\lst{\textbf{\emph{Question $\mbox{\&}$ Answer System:}} Designed competitive Q$\mbox{\&}$A system to attain up to 39.5\% accuracy by detecting Apache Lucene and Natural Language Toolkit, etc.}\\
    Fall 15'& \textbf{Computational Method,} Lehigh University.\\ 
    &\lst{\textbf{\emph{Compressed Sensing:}} Used of $\ell_1$-regularized lasso model to recovery pictures with missing pixels. Multiple algorithms (ISTA, FISTA, GRPS) are implemented in C\texttt{++} and compared}\\   
    % &\lst{\textbf{\emph{Q&A system:}} An competition at Kaggle}\\
    Spring 15'& \textbf{Pattern Recognition,} Lehigh University.\\ 
    & \lst{\textbf{\emph{Digit Recognizer:}} Implemented a Matlab software package to compare various of classifiers (Support Vector Machine, Artificial Neural Network, Decision Tree, KKN) for character-image classification problem}\\
    % Fall 14'& \textbf{Linear Optimization,} Lehigh University. \\
    % Fall 14'& \textbf{Convex Analysis,} Lehigh University. \\
    Fall 14'& \textbf{Integer Programming,} Lehigh University.\\ 
    &\lst{\textbf{\emph{Mixed binary problem solver:}} Implemented a Python software package to address mixed binary programming problem with branch and cut method}\\
    Spring 14'& \textbf{Machine Learning,} Andrew Ng (Stanford University), Coursera.\\
    In progress&\textbf{High Performance Scientific Computing,} Randall J. LeVeque, Coursera.\\
    &\textbf{Machine Learning,} Andrew Ng, Coursera.
\end{longtable}

\section{Presentation}
\begin{longtable}{>{\centering}p{3.3cm}|p{14cm}}
    Nov 15' &\textbf{Dual Free SDCA for Empirical Risk Minimization with Adaptive Probabilities,} NIPS 2015, Montréal, Canada.  \\
    Aug 13' &\textbf{Estimating Portfolio Loss Probabilities with Optimal Risk Loading Coefficients and Fixed Dependency among Obligors,} Siemens Corporation, Corporate Technology, Princeton, US.\\
    Nov 14' &\textbf{Random Coordinate Descent Method on Large-scale Optimization Problems,} Coral Seminar, Lehigh University.
\end{longtable}

\section{Honors and Grants}
\begin{longtable}{>{\centering}p{3.3cm}|p{14cm}}
   Jan 16' - May 16' & Dean’s Doctoral Fellowship, Lehigh University.\\
   Sep 15' - Jan 16' & Dean’s Doctoral Fellowship, Lehigh University.\\
   Sep 14' - Sep 15' & Dean’s Doctoral Assistantship, Lehigh University.\\
   Sep 13' - Jun 14' & First Prize of Excellent Master Scholarship, Nankai University.\\
   Sep 12' - Jun 14' & Fellowship Award, Nankai University.\\
   % Jun 12' & Excellent B.Sci Thesis, Nankai University.\\
   % Nov 11'& National Encouragement Scholarship, Nankai University.\\
   % Sep 08' & Second Prize Winner of National Olympics Contest of Math.
\end{longtable}

% \section{Conferences \& Workshop Attended}
% \begin{longtable}{>{\centering}p{3.3cm}|p{14cm}}
%     ~~~Nov 15'  & INFORMS Annual Meeting, Philadelphia\\
%     ~~~Aug 14'  & MOPTA - Modeling and Optimization: Theory and Applications, Lehigh University.\\
%     ~~~Jun 14'  & International Workshop of Spectral Graph and Hypergraph Theory, Fuzhou University.\\
%     ~~~Nov 13' & The 2nd Sino-German Workshop on Optimization, Chinese Academy of Sciences.\\
%     ~~~Jun 12'  & International Conference on the Spectral Theory of the Tensor, Nankai University.\\
%     ~~~Jun 12' & Senior Workshop on the Spectral Theory of the tensor, Tianjin University.
% \end{longtable}

\section{Reference}
Martin Takáč, Department of Industrial and Systems Engineering, H.S. Mohler Laboratory, Lehigh University, Bethlehem, PA 18015, takac@lehigh.edu.

\end{document}
