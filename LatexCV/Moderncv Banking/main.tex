%% start of file `template.tex'.
%% Copyright 2006-2013 Xavier Danaux (xdanaux@gmail.com).
%
% This work may be distributed and/or modified under the
% conditions of the LaTeX Project Public License version 1.3c,
% available at http://www.latex-project.org/lppl/.


\documentclass[11pt,a4paper,sans]{moderncv}        % possible options include font size ('10pt', '11pt' and '12pt'), paper size ('a4paper', 'letterpaper', 'a5paper', 'legalpaper', 'executivepaper' and 'landscape') and font family ('sans' and 'roman')

% moderncv themes
% \moderncvstyle{oldstyle}                            % style options are 'casual' (default), 'classic', 'oldstyle' and 'banking'
% \moderncvstyle{classic}                            % style options are 'casual' (default), 'classic', 'oldstyle' and 'banking'
% \moderncvstyle{casual}                            % style options are 'casual' (default), 'classic', 'oldstyle' and 'banking'
\moderncvstyle{banking}                            % style options are 'casual' (default), 'classic', 'oldstyle' and 'banking'
\moderncvcolor{blue}                                % color options 'blue' (default), 'orange', 'green', 'red', 'purple', 'grey' and 'black'
%\renewcommand{\familydefault}{\sfdefault}         % to set the default font; use '\sfdefault' for the default sans serif font, '\rmdefault' for the default roman one, or any tex font name
%\nopagenumbers{}                                  % uncomment to suppress automatic page numbering for CVs longer than one page

% character encoding
\usepackage[utf8]{inputenc}                       % if you are not using xelatex ou lualatex, replace by the encoding you are using
%\usepackage{CJKutf8}                              % if you need to use CJK to typeset your resume in Chinese, Japanese or Korean

% adjust the page margins
\usepackage[scale=0.75]{geometry}
%\setlength{\hintscolumnwidth}{3cm}                % if you want to change the width of the column with the dates
%\setlength{\makecvtitlenamewidth}{10cm}           % for the 'classic' style, if you want to force the width allocated to your name and avoid line breaks. be careful though, the length is normally calculated to avoid any overlap with your personal info; use this at your own typographical risks...

% personal data
\name{Xi}{He}
\title{Ph.D. Candidate}                               % optional, remove / comment the line if not wanted
\address{837 Cedar Hill Drive}{Allentown}{Pennsylvania}% optional, remove / comment the line if not wanted; the "postcode city" and and "country" arguments can be omitted or provided empty
\phone[mobile]{+1~(484)~633~8040}                   % optional, remove / comment the line if not wanted
% \phone[fixed]{+2~(345)~678~901}                    % optional, remove / comment the line if not wanted
% \phone[fax]{+3~(456)~789~012}                      % optional, remove / comment the line if not wanted
\email{xih314@lehigh.edu}                               % optional, remove / comment the line if not wanted
\homepage{https://xihey.com}                         % optional, remove / comment the line if not wanted
% \extrainfo{additional information}                 % optional, remove / comment the line if not wanted
% \photo[64pt][0.4pt]{picture}                       % optional, remove / comment the line if not wanted; '64pt' is the height the picture must be resized to, 0.4pt is the thickness of the frame around it (put it to 0pt for no frame) and 'picture' is the name of the picture file
% \quote{Some quote}                                 % optional, remove / comment the line if not wanted

% to show numerical labels in the bibliography (default is to show no labels); only useful if you make citations in your resume
%\makeatletter
%\renewcommand*{\bibliographyitemlabel}{\@biblabel{\arabic{enumiv}}}
%\makeatother
%\renewcommand*{\bibliographyitemlabel}{[\arabic{enumiv}]}% CONSIDER REPLACING THE ABOVE BY THIS

% bibliography with mutiple entries
%\usepackage{multibib}
%\newcites{book,misc}{{Books},{Others}}
%----------------------------------------------------------------------------------
%            content
%----------------------------------------------------------------------------------
\begin{document}
%-----       letter       ---------------------------------------------------------
% recipient data
\recipient{Company Recruitment team}{Siemens Corporate Research.\\720 College Rd. E\\Princeton}
\date{\today}
\opening{Dear Dr. Akrotirianakis}
\closing{Yours faithfully,}
% \enclosure[Attached]{curriculum vit\ae{}}          % use an optional argument to use a string other than "Enclosure", or redefine \enclname
\makelettertitle

I am writing to express my interest in the Predictive Analytics and Monitoring internship at the
Siemens Corporate Research during this summer. I'm currently a second year Ph.D. students in Industrial and System Engineering from Lehigh University, 
and I obtained my MS/BS degree both major in Mathematics before. I am confident that my strong mathematic background and two years 
optimization research experience would allow me to contribute your productive group, as well as explode my career of real-world application with my passion.

Previously, I have had the opportunity to work on several relevant projects that would provide me with
the skill sets I need to be an qualified researcher. These projects have included: explore negative 
curvature of deep learning network to boost its performance, develop well-optimized C\texttt{++} software 
package for image recovery, and implemented Matlab software package to compare various of classifier technologies.

I would be excited to join the Predictive Analytics and Monitoring group. The group has exceptional appeal for me
because of the relevant research topics (e.g. optimization, machine learning) and opportunities (e.g., real-world industrial challenge). 
To be one member of your group, I would relish the opportunity to leverage my past
experiences for this worthy situation, and learn from the experiences of my fellow group members.

Thank you in advance for your time and consideration.  I look forward to the opportunity to speak with you and discuss the
position in more detail.

\makeletterclosing

\end{document}


%% end of file `template.tex'.
